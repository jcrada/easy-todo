\documentclass[a4paper]{article}
\usepackage{listings}
\usepackage[enabled,section]{easy-todo}

\begin{document}
	\title{The \texttt{easy-todo} package}
	\author{Juan Rada-Vilela}
	\date{January, 2011}
	
	\maketitle
	
	\begin{abstract}
		The \texttt{easy-todo} package allows to add TODO notes all along the document and show the list of TODOs as an index with references. 
	\end{abstract}
	
	\section{Options}
	When including the package, the following options are available:
	
	\begin{description}
		\item [enabled] Shows the TODO notes as well as the index.
		\item [disabled] Hides the TODO notes as well as the index. Useful for printing drafts without the TODO notes but keeping them in the document.
		\item [final] Same as disabled.
		\item [chapter] Prints the list of TODOs as a chapter.
		\item [section] Prints the list of TODOs as a section.
	\end{description}
	
	For example:
\lstset{language=TeX}
\begin{lstlisting}
\usepackage[enabled,section]{easy-todo}
\end{lstlisting}

	\section{Commands}
	\begin{description}
		\item [\textbackslash todo\{note\}] Adds in the same location a TODO note with text \texttt{note}.
		\item [\textbackslash listoftodos] Creates the list of TODOs in the same location as used.
	\end{description}
	
	\todo{This is a TODO note}
	
	\todo{I wish there was more to mention about this package, but it is so simple that it is straight-forward to understand the beginner's code}
	
	\listoftodos
\end{document}